\documentclass[11pt]{article}
%Gummi|065|=)
\usepackage[title]{appendix}
\usepackage{hyperref}
\usepackage{amsmath}
\usepackage{graphicx}
\usepackage{epigraph}
\title{\textbf{Euler's Totient Function}}
\date{}
\begin{document}

\maketitle

A Swiss mathematician from 18th century, called Leonhard Euler formulated the following proposition: \\

\textit{If n is a positive integer, than totient function of n gives the amount of integers, that are less than n and relatively prime to n .} \\

This post is dedicated to express the formula and elaborate the proof of it.

\section{Intro}

Totient function is also called as phi function $\phi(n)$. For example, for $n=9$, $\phi(n) = 6$ , since we have 6 numbers - 1, 2, 4, 5, 7 and 8 - that are relatively prime to 9 and less than 9.

 Formula states that, if we factorize our number to it's prime divisors $n = p_{1}^{n_1} ...p_{k}^{n_k}$ then we can calculate totient function of our number n as following: 

\begin{equation}
\phi(n) = \prod\limits_{i} p_{i}^{n_i} (1 - \frac{1}{p_{i}}) = n \prod\limits_{i}(1 - \frac{1}{p_i})
\end{equation}

Let's try this one, when n = 12 . We can factorize $12 = 2^2 \times 3$ and totient function: 

\begin{equation}
\phi(12) = 12 . (1 - \frac{1}{3}) . (1 - \frac{1}{2}) = 12 . \frac{2}{3} . \frac{1}{2} = 4 
\end{equation}

Since we only have four numbers - 1, 5, 7, 11 - that are relatively prime to 12 and less than 12 ; formula is valid so far.

In order to generalize it -after those example based explanations- let the proof begins...

\section{Proof}
Firstly, we would prove that $\phi(p)$ function is multiplicative where, if \\
$p = m \times n$ and $m$ and $n$ are relatively prime to each other,

\begin{equation}
\phi(p) = \phi(m) \times \phi(n)
\end{equation}

Consequently, we would make the conclusion, which surely takes less effort. \\

First, let's write all the numbers in between 1 and $m \times n$ in a matrix, such as: \\


$\begin{bmatrix} 1 & 2 & 3 & \cdots & r & \cdots & m \\ m + 1 & m + 2 & m + 3 & \cdots & m + r & \cdots & 2m \\ \vdots & \vdots & \vdots & \ddots & \cdots & \ddots & \vdots \\ (n - 1)m + 1 & (n - 1)m + 2 & (n - 1)m + 3 & \cdots & (n - 1)m + r & \cdots & mn \end{bmatrix}$\\

An element in this matrix should be relatively prime to both $m$ and $n$, in order to be relatively prime to $p$. In the first row, we have $\phi(m)$ amount of elements, that are relatively prime to $m$. This hold for other rows as well, since $gcd(r, m) = gcd(r + m, m)$ \\

Therefore, we can say that $gcd(r,m) = 1 \rightarrow gcd(r+km, 1) = 1$, if $r$ is relatively prime with $m$ all elements in the column of $r$ is relatively prime with $m$ as well. \\

Now, let's concentrate to find out elements, which are relatively prime to $n$. If we can find out them and if they are in the same column with elements, that are relatively prime to $m$, then we can be sure that this element is relatively prime to $p$ as well.\\

This time, we would go into our matrix column by column. Vertically consecutive elements in each column has difference of m . Since m and n are relatively prime numbers, $gcd(m,n) = 1$ which is quite critical for explanations below.\\

From this point of view, we know that 

\begin{equation}
\begin{split}
\bmod (r, n) &= \bmod(r + mn, n) \\
&= \bmod(r + 2mn, n) \\
&=  \bmod(r + kmn, n) 
\end{split}
\end{equation}

Since the greatest common divisor of m and n is equal to 1, each element in a rows of column would have unique remainders for modulo n. In order to have same remainder of modulo n in different rows of particular column, we need to satisfy: 

\begin{equation}
\bmod(r + km, n) = \bmod(r, n) \text{ where } 0 < k < n. 
\end{equation}

However, adding multiples of relatively prime number, like m, on top of any r, does not give the same remainder for modulo n. So modulo n of all elements in each column is equal to set of numbers from 0 to n-1. Consequently, in each column we have $\phi(n)$ amount of numbers, that are relatively prime to n. \\

As a conclusion for this step, we have $\phi(m)$ amount of numbers in each row, that are relatively prime to m and $\phi(n)$ amount of numbers in each column, that are relatively prime to n and finally we can say that, this matrix has $\phi(m) \times \phi(n)$ amount of elements, that are relatively prime to p. \\

You can check and visualize the proposition above with following matrix, where $m = 9, n = 4$.\\

$\begin{bmatrix} 1 & 2 & 3 & 4 & 5 & 6 & 7 & 8 & 9 \\ 10 & 11 & 12 & 13 & 14 & 15 & 16 & 17 & 18 \\ 19 & 20 & 21 & 22 & 23 & 24 & 25 & 26 & 27 \\ 28 & 29 & 30 & 31 & 32 & 33 & 34 & 35 & 36 \end{bmatrix}$ \\

Now, we are quite close to complete the proof, before that we would have small theorem and proof. \\

\textit{Small Theorem:} If $p$ is a prime number, and $k > 0$ , then 

\begin{equation}
\phi(p^k) = p^k - p^{k-1} = p^k(1 - \frac{1}{p}) 
\end{equation}

\textit{Small Proof:}  Since p is a prime number, the numbers that are not relatively prime with $p^{k}$ has to be divisible with p and we have $p^{k-1}$ amount of them. For example, let p = 7 and k = 2 . We are seeking amount of numbers that are relatively prime with

\begin{equation}
p^{k} = 7^2 = 49 
\end{equation}

A number has to be divisible with 7 in order to not be relatively prime with 49 and we have $\frac{p^k}{p^{k-1}} = \frac{7^2}{7^1} = 7$ of them, which are $1, 7, 14, 21, 28, 35, 42$.
By using multiplicative property of $\phi$ function and this small theoem, here we accomplish the proof.

\begin{equation}
\begin{split}
\phi(n) &= \phi (p_1^{k_1}) \phi (p_2^{k_2}) \text{ . . . } \phi (p_n^{k_n}) \\
&= p_1^{k_1}p_2^{k_2} \text{ . . . } p_n^{k_n} (1 - \frac{1}{p_1}) (1 - \frac{1}{p_2}) \text{ . . . } (1 - \frac{1}{p_n}) \\
&= n(1 - \frac{1}{p_1})(1 - \frac{1}{p_2}) \text{ . . . }(1 - \frac{1}{p_n}) 
\end{split}
\end{equation}

\begin{thebibliography}{99}

\bibitem{a1}
\url{http://www.math.wisc.edu/~josizemore/Notes11(phi).pdf}

\end{thebibliography}

\end{document}
