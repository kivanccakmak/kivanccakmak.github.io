\documentclass[11pt]{article}
%Gummi|065|=)
\usepackage[title]{appendix}
\usepackage{hyperref}
\usepackage{amsmath}
\title{\textbf{Proof on Infinity of Prime Numbers}}
\author{K{\i}van\c{c} \c{C}akmak\\}
\date{}
\begin{document}

\maketitle

In this post, I show that we have infinitely many prime numbers. To do so, I use a proof from 300 BC which provided by Greek Mathematician. The one that people usually call as “Father of Geometry” or Euclid of Alexandria. 

\section{Intro}

Let's write prime numbers in an ascending manner,

\begin{equation}
\begin{split}
p_1 = 2 \\
p_2 = 3 \\
p_3 = 5  \\
 p_4 = 7
\end{split}
\end{equation}

When we check the numbers which are in a pattern of: 

\begin{equation}
\begin{split}
p_1.p_2 + 1 = 2.3 + 1 = 7 \\
 p_1.p_2.p_3 + 1 = 2.3.5 + 1 = 31 \\
p_1.p_2.p_3.p_4 + 1 = 2.3.5.7 + 1 = 211 \\
\end{split}
\end{equation}

we oftenly come up with a prime number.\\

But how can we be sure about primality of number $A = p_1.p_2 \dots p_n + 1$ for all values of $n$ ? 

\section{Proof}
Here we use contradiction technique to check: We intuitively think that $A$ is a prime number, since it empirically is. Now we set up an equation which assumes that $A$ is a non-prime number and try to show that this equation would not hold --hence it contradicts.

Following equation assumes that $A$ is a divisor of some number. \\

We clearly know that, every number -except $1$- is at least multiple of one prime number. \\

For this reason, let's say that $p_r$ is a divisor of $A$ . 

\begin{equation}
 A = p_1.p_2.p_3 \dots p_r \dots p_n + 1 \rightarrow \frac{A}{p_r} = p_1.p_2.p_3 \dots p_{r-1}.p_{r+1} \dots p_n + \frac{1}{p_r}
\end{equation}

Consequently we have 

\begin{equation}
\frac{A}{p_r} - p_1.p_2.p_3 \dots p_{r-1}.p_{r+1} \dots p_n = \frac{1}{p_r} 
\end{equation}

Let's concentrate on this last equation. \\

$\frac{A}{p_r}$ has to be an integer, since we assume $A$ is a multiple of $p_r$. \\

Second term $p_1.p_2.p_3 \dots p_{r-1}.p_{r+1} \dots p_n$ is also integer, nevertheless $\frac{1}{p_r}$ is not. 
So, our non-primality assumption of $A$ was wrong. Consequently, this wrong assumption proves us that, $A$ would always be prime for any value of $n$.
\end{document}
