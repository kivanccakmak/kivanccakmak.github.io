\documentclass[11pt]{article}
%Gummi|065|=)
\usepackage[title]{appendix}
\usepackage{amsmath}
\title{\textbf{Asymptotic Equipartition Property}}
\date{}
\begin{document}

\maketitle

This blog post aims to introduce the concept of typical sequence, and some properties of these typical sequences, which are collectively known as Asymptotic Equipartition Property.

\section{Typicality}

Consider a sequence of $L = 20$ bits emitted by a discrete memoryless source (DMS) with

\begin{equation}
P_{U}(0) = \frac{3}{4},\:\: P_{U}(1) = \frac{1}{4}
\end{equation}
Which one of the following is more likely to have come out from this particular
source?

\begin{equation}
1,1,1,1,1,1,1,1,1,1,1,1,1,1,1,1,1,1,1,1
\end{equation}

\begin{equation}
1,0,1,0,1,0,1,1,0,0,0,1,0,0,0,0,0,0,0,0
\end{equation}

\begin{equation}
0,0,0,0,0,0,0,0,0,0,0,0,0,0,0,0,0,0,0,0
\end{equation}

The first sequence is a all one sequence. The second one has 14 zeros and
6 ones. The third is a all zero sequence. Probability of occurrence of the sequences are $\frac{1}{4}^{20}$, $\frac{1}{4}^{20}\times3^{14}$ and $\frac{1}{4}^{20}\times3^{20}$ respectively. 
\\

All zero and all one sequence are very unlikely to come out of this source
because I have been given that $P_{U}(0) = \frac{3}{4},\:\: P_{U}(1) = \frac{1}{4}$.
\\

If we take large enough block size it is likely that we should get 0 three fourth of time, and we should get 1 one fourth of time. 
\\

If we had $P_{U}(0) = 0.99,\:\: P_{U}(1) = 0.01$, then we can say the all zero sequence is most likely to happen.
\\

The second sequence has 14 zeros and 6 ones. Roughly if we take large enough bits, I expect $\frac{3}{4}$ number of 1s and rest to be zero. Now, we will formally define what is a typical sequence and then we will show some properties of a typical sequence.

\begin{itemize}
\item Let \textbf{U} denote an output sequence of length \textbf{L} emitted a K-ary DMS and $P_{U}(u)$ is the output probability distribution.
\item Let \textbf{u} = [$u_{1}$, $u_{2}$, ..., $u_{L}$] denote possible values of, i.e. $u_{j} = \{{a_{1}, a_{2}, ..., a_{K}}\}$ for $1 \leq j \leq L$.
\item Let $n_{ai}(u)$ denotes the number of occurence of the letter $a_{i}$ in the sequence \textbf{u}. Then \textbf{u} is an $\epsilon-$typical sequence of length \textbf{L} for this K-ary DMS if: \\ \\
$(1 - \epsilon)P_{u}(a_{i}) \leq \frac{n_{ai}(\textbf{u})}{L} \leq (1 + \epsilon) P_{u}(a_{i})$, $1 \leq i \leq K$
\end{itemize}
Please note $\epsilon$ is a small number. We consider the same example as we considered earlier.
\\

Here, consider a binary DMS with $P_{U}(0) = \frac{3}{4},\:\: P_{U}(1) = \frac{1}{4}$ and let's choose an $\epsilon = \frac{1}{3}$. Then a sequence \textbf{u} of length $L = 20$ is $\epsilon$-typical if and only if both

\begin{equation}
\frac{2}{3}.\frac{3}{4} \leq \frac{n_{0}(u)}{20} \leq \frac{4}{3}.\frac{3}{4}
\end{equation}

and

\begin{equation}
\frac{2}{3}.\frac{1}{4} \leq \frac{n_{1}(u)}{20} \leq \frac{4}{3}.\frac{1}{4}
\end{equation}

satisfies. 
\\ \\
Equivalently, \textbf{u} is $\epsilon$-typical if and only if both

\begin{equation}
10 \leq n_{0}(u) \leq 20
\end{equation}

and

\begin{equation}
4 \leq n_{1}(u) \leq 6
\end{equation}


\section{Properties}

\textit{Property 1}: If \textbf{u} is an $\epsilon$-typical output sequence of length \textbf{L} from a K-ary
DMS with entropy $H(U)$ in bits, then:

\begin{equation}
2^{-(1+\epsilon)LH(U)} \leq P_{U}(\textbf{u}) \leq 2^{-(1-\epsilon)LH(U)}
\end{equation}

\textit{Proof:} From the definition of a DMS, we have

\begin{equation}
P_{U}(\textbf{u}) = \prod\limits_{j=1}^L P_{U}(u_{j})
\end{equation}

So, since this is a K-ary source, it will emit $a_{1}, a_{2}, \dots, a_{K}$. So, let us say $a_{1}$ is emitted $n_{a1}(u)$, $a_{2}$ is $n_{a2}(u)$ times, similarly $a_{K}$ is $n_{aK}(u)$ times. 

So, we can write:

\begin{equation}
P_{U}(\textbf{u}) = \prod_{j=1}^LP_{U}(u_{j}) = \prod_{i=1}^K[P_{U}(a_{i})]^{n_{ai}(\textbf{u})}
\end{equation}

Here, we can say that:
\\
\begin{equation}
P_{U}(\textbf{u}) = \prod_{i=1}^K[P_{U}(a_{i})]^{n_{ai}(\textbf{u})} = \prod_{i=1}^K[P_{U}(a_{i})]^{LP_{U}(a_{i})} \geq \prod_{i=1}^K[P_{U}(a_{i})]^{(1+\epsilon)LP_{U}(a_{i})}
\end{equation}

since raising a power of a number in between 0 and 1 (which is probability of $a_{i}$ in this case) with a positive number (which is $\epsilon$ in this case) decreases it's value.
\\ 

Equivalently (see logarithmic exchange part below),
\begin{equation}
P_{U}(\textbf{u}) \geq \prod_{i=1}^K2^{(1+\epsilon)LP_{U}(a_{i})\log_{2}P_{U}(a_{i})}
\end{equation}
\\

Simplifying we get,
\begin{equation}
P_{U}(\textbf{u}) \geq 2^{(1+\epsilon) L \sum\limits_{i=1}^KP_{U}(a_{i})\log_{2}P_{U}(a_{i})}
\end{equation}
\\

Hence
\begin{equation}
P_{U}(\textbf{u}) \geq 2^{-(1+\epsilon)LH(U)}
\end{equation}
\\

Similar arguments can be used to prove
\begin{equation}
P_{U}(\textbf{u}) \leq 2^{-(1-\epsilon)LH(U)}
\end{equation}


\textit{Property 2}: The probability $1 - P(F)$, that the lenght $L$ output sequence $U$ from a K-ary DMS is $\epsilon-$typical satisfies:

\begin{equation}
1 - P(F) \leq 1 - \frac{K}{L\epsilon^{2}P_{min}}
\end{equation}

where $P_{min}$ is the smallest positive value of $P_{U}(u)$

\begin{appendices}
\section{Logarithmic Exchange}
\textit{Lemma:} $b^{\log_{b}(x)} = x$
\\ \\
\textit{Proof:} $log_{b}(x)$ answers \textit{"b to what power equals x?"} via following property.
\begin{equation}
\log_{b}(x) = y \Leftrightarrow b^{y} = x
\end{equation}
\section{Variance and Mean of a Binary Random Variable}

let $P(X=0) = a$ and $P(X=1) = 1 - a$.

\begin{equation} \label{mean_x}
\begin{split}
E[X] & = (1 - P(X=1)) \times 0 + P(X=1) \times 1 = P(X=1) \\
& = (1 - a) \times 0 + a \times 1 \\
& = a \\ \\
\mathrm{Var}[X] & = E[(X-\mu)^2] \\
& = (1 - P(X=1)) \times (0 - P(X=1))^2 + P(X=1) \times (1 - P(X=1))^2\\
& = (1 - a) \times (0 - a)^2 + a \times (1 - a)^2 \\
& = (1 - a) \times (a^2 +a - a^2) \\
& = (1 - a) \times a
\end{split}
\end{equation}


\end{appendices}



\end{document}
