\documentclass[11pt]{article}
%Gummi|065|=)
\title{\textbf{Asymptotic Equipartition Property}}
\date{}
\begin{document}

\maketitle

This blog post aims to introduce the concept of typical sequence, and some properties of these typical sequences, which are collectively known as Asymptotic Equipartition Property.

\section{Typicality}

Consider a sequence of $L = 20$ bits emitted by a discrete memoryless source (DMS) with

\begin{equation}
P_{U}(0) = \frac{3}{4},\:\: P_{U}(1) = \frac{1}{4}
\end{equation}
Which one of the following is more likely to have come out from this particular
source?

\begin{equation}
1,1,1,1,1,1,1,1,1,1,1,1,1,1,1,1,1,1,1,1
\end{equation}

\begin{equation}
1,0,1,0,1,0,1,1,0,0,0,1,0,0,0,0,0,0,0,0
\end{equation}

\begin{equation}
0,0,0,0,0,0,0,0,0,0,0,0,0,0,0,0,0,0,0,0
\end{equation}

The first sequence is a all one sequence. The second one has 14 zeros and
6 ones. The third is a all zero sequence. Probability of occurrence of the sequences are $\frac{1}{4}^{20}$, $\frac{1}{4}^{20}\times3^{14}$ and $\frac{1}{4}^{20}\times3^{20}$ respectively. 
\\

All zero and all one sequence are very unlikely to come out of this source
because I have been given that $P_{U}(0) = \frac{3}{4},\:\: P_{U}(1) = \frac{1}{4}$.
\\

If we take large enough block size it is likely that we should get 0 three fourth of time, and we should get 1 one fourth of time. 
\\

If we had $P_{U}(0) = 0.99,\:\: P_{U}(1) = 0.01$, then we can say the all zero sequence is most likely to happen.
\\

The second sequence has 14 zeros and 6 ones. Roughly if we take large enough bits, I expect $\frac{3}{4}$ number of 1s and rest to be zero. Now, we will formally define what is a typical sequence and then we will show some properties of a typical sequence.

\begin{itemize}
\item Let \textbf{U} denote an output sequence of length \textbf{L} emitted a K-ary DMS and $P_{U}(u)$ is the output probability distribution.
\item Let \textbf{u} = [$u_{1}$, $u_{2}$, ..., $u_{L}$] denote possible values of, i.e. $u_{j} = \{{a_{1}, a_{2}, ..., a_{K}}\}$ for $1 \leq j \leq L$.
\item Let $n_{ai}(u)$ denotes the number of occurence of the letter $a_{i}$ in the sequence \textbf{u}. Then \textbf{u} is an $\epsilon-$typical sequence of length \textbf{L} for this K-ary DMS if: \\ \\
$(1 - \epsilon)P_{u}(a_{i}) \leq \frac{n_{ai}(\textbf{u})}{L} \leq (1 + \epsilon) P_{u}(a_{i})$, $1 \leq i \leq K$
\end{itemize}
Please note $\epsilon$ is a small number. We consider the same example as we considered earlier.

\end{document}
